\documentclass[a4paper]{article}
%\usepackage{fullpage}
%\usepackage{parskip}
\usepackage{amsmath}
\usepackage{amsfonts}
\usepackage{amssymb}
\usepackage{amsthm}
\usepackage{graphicx}
\usepackage{color}
\usepackage{float}
\usepackage{url}
\usepackage{hyperref}
\usepackage{tikz}
\hypersetup{ colorlinks=true, urlcolor=blue}

\usepackage{datetime}

\sloppy
\definecolor{lightgray}{gray}{0.5}
\setlength{\parindent}{0pt}


\usepackage[sc]{mathpazo}
\linespread{1.05}         % Palatino needs more leading (space between lines)
\usepackage[T1]{fontenc}
\usepackage{pxfonts}


%------------------------------------------------------------

\begin{document}
\title{The Aleron Performance Dashboard}
%\\Requirements, Design and Challenges}
\author{Mohsin Javed}
\date{July 16, 2015\footnote{Last Modified: \currenttime, \today }}
%Version: \version
\maketitle
\section*{The Aleron Performance Dashboard}
Our dashboard will be a tool available on the web for
quantitative and qualitative analysis of key performance
indicators (KPIs) and data provided by our clients. The aim
of the dashboard is to provide an automated way for our
clients to upload their data on the web and to visualize it,
again on the web with key and critical factors of the data
highlighted via the analysis we perform.


\section*{Technologies Used}
There are 4 state of the art key technologies we are using
to build our dashboard:
\begin{itemize}
\item
\href{https://en.wikipedia.org/wiki/Bootstrap_(front-end_framework)}{\textbf{Bootstrap}}:
Developed and used by twitter. It is a tool
for designing front end of responsive websites. A responsive
website looks pretty whether one views it on a huge desktop
screen or on a smart phone. As of 2015, Bootstrap is the
most starred project on \href{https://github.com/}{GitHub}.
This basically means that if there were a facebook of the
tech world, Bootstrap would have the most likes!
\item \href{http://uk.mathworks.com/}{\textbf{MATLAB}}: Arguably, the
best tool for technical computing. Ideal for performing
quick analysis on all kinds of data and comes with a lot of
built in mathematical power. We will use MATLAB to perform
key quantitative analysis on the data provided by our
clients.
\item
\href{https://en.wikipedia.org/wiki/JavaScript}{\textbf{JavaScript}}: 
The language of the web. Whenever we go
online, we almost surely use JavaScript. JavaScript enables
the web browser to perform various actions and tasks. Our
dashboard will use JavaScript to receive data and perform
the necessary processing before it can be visualized. 
\item \href{http://d3js.org}{\textbf{d3}}: The coolest tool out there
for data visualization. d3 stands for data driven documents
and the name says it all. \href{http://d3js.org}{d3} is also 
used by the New York Times for creating rich graphics and
visualisation of all sorts. We will use this tool for
creating the real magic for our dashboard.
\end{itemize}


\section*{How it works}
Here is a mock example of how the dashboard will (should)
work.

\subsection*{Key Steps}
\begin{enumerate}

\item Our client logs in on our dashboard website and
provides us two lists:
\begin{itemize}
   \item A list of KPI, i.e.\ the names of all the key
   performance indicators.  
   \item A list of names and email addresses identifying
   people responsible for each of the  KPIs.
\end{itemize}

\item Based on the lists provided above, we will then
automatically prepare excel sheets with the KIP headings and
email these to the corresponding email addresses. 

\item These individuals will then (hopefully) send back their excel files
populated with relevant data. 

\item Once all data is received, it will be processed by our
analytic engine designed in MATLAB.

\item Processed data will then be picked up and sent on the
web using JavaScript.

\item Our data visualization tool (d3) will be ready to
display it on the web.

\item We will then notify our client to sit back, relax and
enjoy the show.
\end{enumerate}

\bibliographystyle{plain}
\bibliography{../../../../citations/bibliobooks,../../../../citations/bibliopapers}
\end{document}

